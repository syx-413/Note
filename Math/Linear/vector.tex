
\section{向量}
\subsection{线性相关性}

使用行列式不等于$0$的方法最方便,但是有时候行列不同就不能这么做了。

\paragraph{\boxed{\text{初等运算}}}

多用于选择题,给出$n$维线性无关向量,判断向量组是否线性无关。如果向量组初等运算为0就代表线性相关。

\textbf{例题:}已知$n$维向量$\alpha_1$,$\alpha_2$,$\alpha_3$线性无关,则判断线性相关性:$\alpha_1+\alpha_2$,$\alpha_2-\alpha_3$,$\alpha_3+\alpha_1$。

解:$\alpha_1+\alpha_2$与$\alpha_2-\alpha_3$,共同出现了$\alpha_2$,首先要消掉$\alpha_2$,所以相减得到$\alpha_1+\alpha_3$,然后发现跟后面的$\alpha_3+\alpha_1$一样,所以直接一减得到0,表示线性相关。

\paragraph{\boxed{\text{定义法}}}

基本是证明题,若证明$\alpha$、$\beta$线性无关,则令$k_1\alpha+k_2\beta=0$,判断$k_i$的值,如果只有零解则代表$k$矩阵为满秩,从而线性无关。

\subsubsection{代入重组}

若要求线性相关的式子由其他向量构成,则将式子代入表示目标式子。

\textbf{例题:}设$\alpha_1$,$\alpha_2$,$\beta_1$,$\beta_2$,$\beta_3$都是$n$维向量,$n\geqslant3$,且$\beta_1=\alpha_1+\alpha_2$,$\beta_2=\alpha_1-2\alpha_2$,$\beta_3=3\alpha+1+2\alpha_2$,证明向量组$\beta_1$,$\beta_2$,$\beta_3$线性相关。

证明:若存在$k_1,k_2,k_3$使得$k_1\beta_1+k_2\beta_2+k_3\beta_3=0$。

代入$\alpha$表示$\beta$的式子:$k_1(\alpha_1+\alpha_2)+k_2(\alpha_1-2\alpha_2)+k_3(3\alpha_1+2\alpha_2)=0$。

$\therefore(k_1+k_2+3k_3)\alpha_1+(k_1-2k_2+2k_3)\alpha_2=0$。

$\therefore k_1+k_2+3k_3=0$,且$k_1-2k_2+2k_3=0$即可。

而未知数的个数大于方程个数,所以有无穷多解,从而必然有非零解,从而$\beta_1$,$\beta_2$,$\beta_3$线性相关。

\subsubsection{同乘}

若要求线性相关的式子存在一定的乘积关系,则可以用同乘一步步消去系数。

\textbf{例题:}设$A$是$n$阶矩阵,若存在正整数$k$,使得线性方程组$A^kx=0$有解向量$\alpha$,且$A^{k-1}\alpha\neq0$,证明向量组$\alpha,A\alpha,\cdots,A^{k-1}\alpha$线性无关。

证明:假设$\alpha,A\alpha,\cdots,A^{k-1}\alpha$线性相关,则设存在系数$\lambda_1,\lambda_2,\cdots,\lambda_k$使得$\lambda_1\alpha+\lambda_2A\alpha+\cdots+\lambda_kA^{k-1}\alpha=0$。

$\because A^kx=0$的解为$\alpha$,$\therefore A^k\alpha=0$,$\therefore\cdots=A^{k+2}\alpha=A^{k+1}\alpha=A^k\alpha=0$。

左乘$A^{k-1}$,得到$\lambda_1A^{k-1}\alpha+\lambda_2A^k\alpha+\cdots+\lambda_kA^{2k-2}\alpha=\lambda_1A^{k-1}\alpha=0$。

$\because A^{k-1}\alpha\neq0$,$\therefore\lambda_1=0$,消去$\lambda_1$:$\lambda_2A\alpha+\lambda_3A^2\alpha+\cdots+\lambda_kA^{k-1}\alpha=0$。

左乘$A^{k-2}$,得到$\lambda_2A^{k-1}\alpha+\lambda_3A^k\alpha+\cdots+\lambda_kA^{2k-3}\alpha=\lambda_2A^{k-1}\alpha=0$。

$\because A^{k-1}\alpha\neq0$,$\therefore\lambda_2=0$,消去$\lambda_2$:$\lambda_3A^2\alpha+\lambda_4A^3\alpha+\cdots+\lambda_kA^{k-1}\alpha=0$。

同理依次左乘$A^n$,所以$\lambda_1=\lambda_2=\cdots=\lambda_k=0$,所以$\alpha,A\alpha,\cdots,A^{k-1}\alpha$线性无关。

\paragraph{\boxed{\text{行列式}}}

对向量的线性相关性可以从其向量组组成的行列式来计算,若行列式值为0则线性相关,若行列式值不为0则线性无关。

注意这里容易失根。要仔细找出所有为0的因式,不要随便降低阶数。

\textbf{例题:}设$a_1,a_2,\cdots,a_s$是$s$个互不相同的数,探究$s$个$n$维列向量$\alpha_i=[1,a_i,a_i^a,\cdots,a_i^{n-1}]^T$($i=1,2,\cdots,s$)的线性相关性。

解:当$s>n$时,有$n$个方程$s$个未知数,所以必然存在自由变量,从而必然线性相关性。

当$s=n$时,$\vert\alpha_1 \alpha_2 \cdots \alpha_n\vert=\left|\begin{array}{cccc}
    1 & 1 & \cdots & 1 \\
    a_1 & a_2 & \cdots & a_n \\
    \vdots & \vdots & \ddots & \vdots \\
    a_1^{n-1} & a_2^{n-1} & \cdots & a_n^{n-1}
\end{array}\right|=\prod\limits_{1\leqslant j\leqslant i\leqslant n}(a_i-a_j)\neq0$。所以线性无关。

当$s<n$时,对方程矩阵切割保留方形的$s$个$=\left|\begin{array}{cccc}
    1 & 1 & \cdots & 1 \\
    a_1 & a_2 & \cdots & a_n \\
    \vdots & \vdots & \ddots & \vdots \\
    a_1^{n-1} & a_2^{n-1} & \cdots & a_n^{n-1} \\
    \vdots & \vdots & \vdots & \vdots
\end{array}\right|$,上面因为范德蒙德行列式已经不等于0,即上面的方阵线性无关,原来无关延长无关,所以整个方程都线性无关。

综上当$s>n$时线性相关,$s\leqslant n$时线性无关。

\paragraph{\boxed{\text{矩阵秩}}}

当向量的个数与维数不同时就不能使用行列式去分析,而只能用矩阵的秩来分析。当矩阵满秩则线性无关,当矩阵降秩则线性相关。

当谈到多个向量是否线性相关时可以将向量组组成矩阵,判断其秩。满秩就是线性无关,降秩就是线性相关。

当谈到一个向量是否能被其他向量线性表出时,要将这些向量全部组成一起,判断能否被其他向量表出的向量放在最右边,然后判断增广矩阵的秩。

\begin{enumerate}
    \item 若$r(\alpha_1,\alpha_2,\cdots)\neq r(\alpha_1,\alpha_2,\cdots,\beta)$,则$\beta$无法被$\alpha$线性表出。
    \item 若$r(\alpha_1,\alpha_2,\cdots)=r(\alpha_1,\alpha_2,\cdots,\beta)<r$,则$\beta$可以被$\alpha$无穷线性表出。表达式为基础解系。
    \item 若$r(\alpha_1,\alpha_2,\cdots)=r(\alpha_1,\alpha_2,\cdots,\beta)=r$,则$\beta$可以被$\alpha$惟一线性表出。表达式为将矩阵化为单位矩阵后$\beta$所在就是$\alpha$的系数。
\end{enumerate}

\textbf{例题:}已知$\alpha_1=(1,2,1)^T$,$\alpha_2(2,3,a)^T$,$\alpha_3=(1,a+2,-2)^T$,$\beta=(1,3,0)^T$,若$\beta$可以由$\alpha_1$、$\alpha_2$,$\alpha_3$线性表示,且表示法不唯一,求$a$。

解:设$x_1\alpha_1+x_2\alpha_2+x_2\alpha_3=\beta$,由$\beta$可以由$\alpha_1$、$\alpha_2$,$\alpha_3$线性表示,且表示法不唯一可知$Ax=\beta$有无穷解,即$r(A)=r(A|B)<3$。

$=[\alpha_1,\alpha_2,\alpha_3|\beta]=\left[\begin{array}{cccc}
    1 & 2 & 1 & 1 \\
    2 & 3 & a+2 & 3 \\
    1 & a & -2 & 0
\end{array}\right]=\left[\begin{array}{cccc}
    1 & 2 & 1 & 1 \\
    0 & -1 & a & 1 \\
    0 & 0 & a^2-2a-3 & a-3
\end{array}\right]$。

$\therefore a=3$。

\subsection{线性表出}

\paragraph{\boxed{\text{极大线性无关组}}}

极大线性无关组一般与向量组秩在一起使用。一般解出极大线性无关组与秩,还要用极大线性无关组表示出其余的向量,基本步骤:

\begin{enumerate}
    \item 将向量组拼接为矩阵$A$,对$A$进行初等行变换,化为最简行阶梯形矩阵,确定矩阵秩$r(A)$。
    \item 在最简行阶梯矩阵中按列找出一个秩为$r(A)$的子矩阵,即在每个台阶上找一列列向量,找$r(A)$列构成一个新矩阵,其就是一个极大线性无关组。
    \item 将其余向量依次与极大线性无关组进行对比解出表示方法。
\end{enumerate}

\textcolor{orange}{注意:}求向量组的秩可以进行初等变换,包括行变换和列变换。但是求极大线性无关组时最好只使用行变换,因为列变换会改变方程的解。从而解方程组只能做行变换。

\paragraph{\boxed{\text{向量组线性标出}}}

若对于多个向量组成的向量组$B$是否能线性表出向量组$A$(而不是单个向量$\alpha$),把$A$和$B$合并,则若合并后的向量组$C$的秩大于$B$的,那么向量组$B$不能线性表示向量组$A$。

解决方法跟单个向量表出一样,将$B$和$A$合并为增广矩阵,然后进出行变换。

也给出这样的结论,若$B$自身线性相关,则无法线性表出其他矩阵。

\textbf{例题:}设向量组$\alpha_1=(1,0,1)^T$、$\alpha_2=(0,1,1)^T$、$\alpha_3=(1,3,5)^T$不能由向量组$\beta_1=(1,1,1)^T$、$\beta_2=(1,2,3)^T$、$\beta_3=(3,4,a)^T$线性表示,求$a$。

解:已知题目,则$(\beta_1,\beta_2,\beta_3)$线性相关。

对其行变换,解得$a=5$。

\paragraph{\boxed{\text{向量线性表示}}}

即要求$\beta$关于$\alpha_i$的线性表出表达式。

基本方法是设$\beta=a\alpha_1+b\alpha_2+\cdots$,然后每行代入求出,不过也可以使用矩阵变换法。

可以同时求多个$\beta$的表示方式,设$\alpha_i$为长度为$h$的列向量,一共有$n$个,组成$A=(\alpha_1,\alpha_2,\cdots,\alpha_n)$,设$\beta_i$为长度为$h$的列向量,一共有$n$个,组成$B=(\beta_1,\beta_2,\cdots,\beta_n)$。

$[A|B]$通过线性变换得到$[E|C]$,则$B=CA$。

\textbf{例题:}用$\alpha_1=(1,1,1)^T$、$\alpha_2=(1,2,4)^T$、$\alpha_3=(1,3,9)^T$表示$\beta=(1,1,3)^T$。

解:组成矩阵$\left[\begin{array}{cccc}
    1 & 1 & 1 & 1 \\
    1 & 2 & 3 & 1 \\
    1 & 4 & 9 & 3
\end{array}\right]=\left[\begin{array}{cccc}
    1 & 0 & 0 & 2 \\
    0 & 1 & 0 & -2 \\
    0 & 0 & 1 & 1
\end{array}\right]$,所以$\beta=2\alpha_1-2\alpha_2+\alpha_3$。

\subsection{等价向量组}

$r(A)=r(B)=r(A|B)$,所以需要计算三个向量组构成的矩阵的秩就可以了。

\textbf{例题:}设向量组$\alpha$:$\alpha_1=[1,0,2]^T$,$\alpha_2=[0,1,1]^T$,$\alpha_3=[2,-1,a+4]^T$,向量组$\beta$:$\beta_1=[1,2,4]^T$,$\beta_2=[1,-1,a+2]^T$,$\beta_3=[3,3,10]^T$。

矩阵$A=\left(\begin{array}{ccc}
    1 & 0 & 2 \\
    0 & 1 & -1 \\
    2 & 1 & a+4
\end{array}\right)$,$B=\left(\begin{array}{ccc}
    1 & 1 & 3 \\
    2 & -1 & 3 \\
    4 & a+2 & 10
\end{array}\right)$。\medskip

(1)$AB$是否等价。

(2)向量组$AB$是否等价。

(1)解:化简$A=\left(\begin{array}{ccc}
    1 & 0 & 2 \\
    0 & 1 & -1 \\
    0 & 0 & a+1
\end{array}\right)$,$B=\left(\begin{array}{ccc}
    1 & 3 & 1 \\
    0 & 1 & 1 \\
    0 & 0 & a
\end{array}\right)$

若$a\neq-1$,则$r(A)=3$,且$a\neq0$,则$r(B)=3$,此时$AB$等价。

若$a=-1$,则$r(A)=2$,$r(B)=3$,$AB$不等价。

若$a=0$,则$r(B)=2$,$r(A)=2$,$AB$不等价。

(2)解:因为向量组$\alpha$拼接在一起就是$A$,$\beta$拼接在一起就是$B$,所以$r(\alpha)=r(A)$,$r(\beta)=r(B)$,$r(\alpha|\beta)=r(A|B)$。

将$AB$拼在一起做行变换,得到$(A|B)=\left(\begin{array}{c:c}
    \begin{matrix}
        1 & 0 & 2 \\
        0 & 1 & -1 \\
        0 & 0 & a+1
    \end{matrix}&
    \begin{matrix}
        1 & 1 & 3 \\
        2 & -1 & 3 \\
        0 & a+1 & 1
    \end{matrix}
\end{array}\right)$。\medskip

若$a\neq-1\neq0$,则$r(A)=r(B)=r(A|B)$。向量组等价。

若$a=-1$或$a=0$,则$r(A)\neq r(B)$,所以不等价。

\subsection{向量空间}

% \textbf{例题:}设$R^3$中有两个基$A$:$\alpha_1=[1,1,0]^T$,$\alpha_2=[0,1,1]^T$,$\alpha_3=[1,0,1]^T$,基$B$:$\beta_1=[1,0,0]^T$,$\beta_2=[1,1,0]^T$,$\beta_3=[1,1,1]^T$。

% (1)求基$B$到基$A$的过渡矩阵。

% (2)已知$\xi$在基$B$下的坐标为$[1,0,2]^T$,求$\xi$在基$A$下的坐标。

% (1)解:过渡矩阵为$A=BC$,即$B^{-1}A=C$。

% (2)解:令在基$A$下的坐标为$(x_1,x_2,x_3)^T$。

% $\therefore\xi=A(x_1,x_2,x_3)^T=B(1,0,2)^T$,$(x_1,x_2,x_3)^T=A^{-1}B(1,0,2)^T$。

\paragraph{\boxed{\text{基坐标}}}

对于任意向量$\alpha=\xi_ix_i=\eta_iy_i$,$\xi_i$、$\eta_i$为基,$x_i$、$y_i$为向量基$\xi_i$、$\eta_i$下的坐标。

\paragraph{\boxed{\text{过渡矩阵}}}

对于两个基$\eta_i$、$\xi_i$,$\eta_i=\xi_iC$的$C$为$\xi_i$到$\eta_i$的过渡矩阵,该式子为基变换公式。

所以得到$x=Cy$,这个公式为坐标变换公式。
