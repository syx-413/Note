\documentclass{note}

\usepackage{mypackage}

\usepackage{comment}
\usepackage{array}
\usepackage{amsmath}

\usepackage{arydshln}
\usepackage{color}
\usepackage{rotating}
\newcommand{\underwrite}[3][]{\genfrac{}{}{#1}{}{\textstyle #2}{\scriptstyle #3}}
\newcommand{\xdownarrow}[1]{{\left\downarrow\vbox to #1{}\right.\kern-\nulldelimiterspace}}
\newcommand{\xuparrow}[1]{{\left\uparrow\vbox to #1{}\right.\kern-\nulldelimiterspace}}

\renewcommand{\thefootnote}{\fnsymbol{footnote}}

\title{数学笔记}
\author{413}
\date{{\builddatemonth\today}\protect\footnote{\text{Build \builddate\today}}}%加了build

\begin{document}

\maketitle

\renewcommand{\thefootnote}{\arabic{footnote}}
\setcounter{footnote}{0}
\setcounter{tocdepth}{2}%设置深度

\tableofcontents

\newpage


%高等数学
\input{Ad_Math/limit}
\input{Ad_Math/function}
\input{Ad_Math/differentiation-of-functions-of-single-variable}
\input{Ad_Math/integal-of-functions-of-single-variable}
\section{向量代数}

\subsection{空间解析几何}

\paragraph{平面方程}

\paragraph{直线方程}

\paragraph{位置关系}

\paragraph{空间曲线}

\subsubsection{投影}

\paragraph{空间曲面}

\subsection{场论初步}

\paragraph{方向导数}

\paragraph{梯度}

\paragraph{散度与旋度}

直接代入公式。

\textbf{例题:}计算向量场$u(x,y,z)=xy^2i+ye^xj+x\ln(1+z^2)k$在点$P(1,1,0)$的散度和旋度。

解:所以$u(x,y,z)=(P,Q,R)$,$P=xy^2$,$Q=ye^x$,$R=x\ln(1+z^2)$。

$\dfrac{\partial P}{\partial x}=y^2$,$\dfrac{\partial Q}{\partial y}=e^x$,$\dfrac{\partial R}{\partial z}=\dfrac{2zx}{1+z^2}$。

代入$P(1,1,0)$,得到散度$\textrm{div}\,\vec{u}=1+e$。

旋度$\overrightarrow{\textrm{rot}}\,\vec{u}=\left\vert\begin{array}{ccc}
    \vec{i} & \vec{j} & \vec{k} \\
    \dfrac{\partial}{\partial x} & \dfrac{\partial}{\partial y} & \dfrac{\partial}{\partial z} \\
    xy^2 & ye^x & x\ln(1+z^2)
\end{array}\right\vert=\dfrac{\partial x\ln(1+z^2)}{\partial y}\vec{i}+\dfrac{\partial xy^2}{\partial z}\vec{j}+\dfrac{\partial ye^x}{\partial x}\vec{k}-\dfrac{\partial xy^2}{\partial y}\vec{k}-\dfrac{\partial ye^x}{\partial z}\vec{i}-\dfrac{\partial x\ln(1+z^2)}{\partial x}\vec{j}=0+0+ye^x\vec{k}-2xy\vec{k}-0-\ln(1+z^2)\vec{j}=-\ln(1+z^2)\vec{j}+(ye^x-2xy)\vec{k}=(0,0,e-2)$。

\input{Ad_Math/differential-calculus-of-multivariate-functions}
\input{Ad_Math/integral-calculus-of-multivariate-functions}
\input{Ad_Math/infinite-series}
\input{Ad_Math/differential-equation}

\newpage

%线性代数
\section{↓以下为线性代数部分↓}
\input{Linear/determinant}
\input{Linear/matrix}
\input{Linear/vector}
\input{Linear/linear-equations-system}
\input{Linear/similar}
\input{Linear/quadratic-form}

%概率论
\input{Probality/random-events-and-probability}
\input{Probality/random-variables-and-distribution}
\input{Probality/digital-features}
\section{大数定律}

\section{中心极限定理}
\input{Probality/mathematical-statistics}


\end{document}
%\begin{comment},\end{comment}
% key  theorem analysis partlist multicols myalgorithm 
% proposition lable \newtheorem 提供定理环境